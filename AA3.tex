%!BIB program = bibtex
\documentclass[12pt]{article}
 
\usepackage[margin=1in]{geometry} 
\usepackage{amsmath,amsthm,amssymb,mathtools}
\usepackage{enumerate}
\usepackage[vlined, ]{algorithm2e} %ruled vlined

 
\newcommand{\N}{\mathbb{N}}
\newcommand{\Z}{\mathbb{Z}}
\renewcommand{\thealgocf}{}

\newenvironment{theorem}[2][Theorem]{\begin{trivlist}
    
\item[\hskip \labelsep {\bfseries #1}\hskip \labelsep {\bfseries #2.}]}{\end{trivlist}}
\newenvironment{lemma}[2][Lemma]{\begin{trivlist}
\item[\hskip \labelsep {\bfseries #1}\hskip \labelsep {\bfseries #2.}]}{\end{trivlist}}
\newenvironment{exercise}[2][Exercise]{\begin{trivlist}
\item[\hskip \labelsep {\bfseries #1}\hskip \labelsep {\bfseries #2.}]}{\end{trivlist}}
\newenvironment{problem}[2][Problem]{\begin{trivlist}
\item[\hskip \labelsep {\bfseries #1}\hskip \labelsep {\bfseries #2.}]}{\end{trivlist}}
\newenvironment{question}[2][Question]{\begin{trivlist}
\item[\hskip \labelsep {\bfseries #1}\hskip \labelsep {\bfseries #2.}]}{\end{trivlist}}
\newenvironment{corollary}[2][Corollary]{\begin{trivlist}
\item[\hskip \labelsep {\bfseries #1}\hskip \labelsep {\bfseries #2.}]}{\end{trivlist}}
\newenvironment{solution}[2][Solution]{\begin{trivlist}
\item[\hskip \labelsep {\bfseries #1}\hskip \labelsep {\bfseries #2.}]}{\end{trivlist}}
\begin{document}

% --------------------------------------------------------------
%                         Start here
% --------------------------------------------------------------

\title{Problem Set 3}%replace X with the appropriate number
\author{Chen Hao \\ MG20330007\\ %replace with your name
    Advanced Algorithms (Fall 2020)} %if necessary, replace with your course title

\maketitle

\begin{solution}{1}
    ~

    The greedy algorithm is as follow:

    \begin{figure}[ht]
        \centering
        \begin{minipage}{.7\linewidth}
            \begin{algorithm}[H]
                \caption{greedy maximum coverage}
                \KwIn{sets $S_1,S_2,\cdots,S_k$}
                initially, $U=\bigcup_{i=1}^mS_i$ and $C=\emptyset$\;
                $i = 1$ \;
                maxcover = 0\;
                \While{$i \le k$}{
                    find $i\in \{1,2,\cdots, m\}$with the largest $|S_i \cap U|$ \;
                    maxcover += $|S_i \cap U|$\;
                    let $C = C \cup \{i\}$ and $U = U \backslash S_i$\;
                }
                \Return{maxcover}
            \end{algorithm}
        \end{minipage}
    \end{figure}
    let $C^* \subseteq \{1, 2, \cdots, m\}, |C| = k$ denote an optimal set cover. Then
    \[OPT = \Big|\bigcup_{i\in C^*}S_i\Big|\]

    By averaging principle, there must be an $S_i$ of size
    \[|S_i| \geq \frac{OPT}k \]

    By the greediness of the algorithm, in the first iteration the algorithm must choose a set $S_i$ of at least this size to add to the set cover $C$.
    Let $x_i$ denote at iteration $i$, the addtion to the maxcover, so $SOL =\sum_{i=1}^kx_i$. Let $U_t$ denote the universe after $t$ iteraton. $U_0 = U$

    After t iteration, with the $m$ subsets $U_t \cap S_1, \cdots, U_t \cap S_m$, and remaining $|U|-\sum_{i=1}^t x_i$ elements, the maximum coverage
    what the smaller instance can achieve become $OPT-\sum_{i=1}^t x_i$. In the first iteration of the smaller instance, we must choose a set at least the
    size $(OPT-\sum_{i=1}^t x_i)/{k}$. Back to the original setting, what we choose must better than that (since the optimal solution for the original
    instance must also be an optimal solution for this smaller instance). So we have
    \[x_{t+1} \geq \frac{OPT-\sum_{i=1}^t x_i}{k} \]

    Let $SOL_t = \sum_{i=1}^t x_i$, we obtain
    \[SOL_{t+1}-SOL_t \geq \frac{OPT-SOL_t}k\]
    that is
    \[OPT - SOL_{t+1} \le (1-\frac1k)(OPT-SOL_t)\]
    then, we can dedcue:
    \[OPT-SOL=OPT-SOL_k \le (1-\frac1k)^k(OPT-SOL_0)=(1-\frac1k)^kOPT\]
    what is
    \[SOL \geq (1-(1-\frac1k)^k)OPT > (1-1/e)OPT\]

\end{solution}

\begin{solution}{2}
    ~

    \begin{itemize}
        \item From the randomized rounding, we have that $\mathrm{Pr}[x_v = 1] = x_v^*$.
              Because
              \begin{align*}
                  \mathrm{Pr}[\hat{y}_{u,v}=1] & =\mathrm{Pr}[\hat{x}_u = 1, \hat{x}_v = 0]                \\
                                               & =\mathrm{Pr}[\hat{x}_u = 1](1-\mathrm{Pr}[\hat{x}_v = 1]) \\
                                               & =x_u^*(1-x_v^*)                                           \\
                                               & \ge (y_{u,v}^*)^2
              \end{align*}

              Addtionaly,
              \begin{align*}
                  OPT \le OPT_{LP} & = \sum_{(u,v) \in E}y_{u, v}^*                                           \\
                                   & \le |E|\sqrt{\frac{\sum_{(u,v) \in E}(y_{u, v}^*)^2}{|E|}}               \\
                                   & \le |E|\sqrt{\frac{\sum_{(u,v) \in E}\mathrm{Pr}[\hat{y}_{u,v}=1]}{|E|}} \\
                                   & = |E|\sqrt{\frac{\mathbb{E}[\sum_{(u,v) \in E}\hat{y}_{u,v}]}{|E|}}      \\
                                   & = |E|\sqrt{\frac{SOL}{|E|}}
              \end{align*}
              So, we obtain
              \begin{align*}
                  \frac{SOL}{OPT} \ge \frac{OPT}{|E|}
              \end{align*}
              Then the approximation ratio is $\frac{OPT}{|E|}$
        \item Same from above, we have
              \begin{align*}
                  \mathrm{Pr}[\hat{y}_{u,v}=1] & =\mathrm{Pr}[\hat{x}_u = 1, \hat{x}_v = 0]                       \\
                                               & =\mathrm{Pr}[\hat{x}_u = 1](1-\mathrm{Pr}[\hat{x}_v = 1])        \\
                                               & = (\frac{1}{4}+\frac{x_u^*}{2})(1-(\frac{1}{4}+\frac{x_u^*}{2})) \\
                                               & \ge (\frac{1}{4}+\frac{y_{u,v}^*}{2})^2                          \\
                                               & = \frac{(y_{u,v}^*)^2}{4} + \frac{y_{u,v}^*}{4} + \frac{1}{16}   \\
                                               & = \frac{(y_{u,v}^*)^2}{4} + \frac{1}{16} + \frac{y_{u,v}^*}{4}   \\
                                               & \ge 2\sqrt{\frac{(y_{u,v}^*)^2}{64}} + \frac{y_{u,v}^*}{4}       \\
                                               & = \frac{y_{u,v}^*}{2}
              \end{align*}
              So, we obtain
              \begin{align*}
                  \frac{SOL}{OPT} & \ge \frac{SOL}{OPT_{LP}}                                                             \\
                                  & = \frac{\mathbb{E}[\sum_{(u,v) \in E}\hat{y}_{u,v}]}{\sum_{(u,v) \in E}y_{u,v}^*}    \\
                                  & = \frac{\sum_{(u,v) \in E}\mathrm{Pr}[\hat{y}_{u,v}=1]}{\sum_{(u,v) \in E}y_{u,v}^*} \\
                                  & \ge \frac{\sum_{(u,v) \in E}\frac{y_{u,v}^*}2}{\sum_{(u,v) \in E}y_{u,v}^*}          \\
                                  & = 0.5
              \end{align*}
              Then the approximation ratio is 0.5
    \end{itemize}
\end{solution}

\begin{solution}{3}
    ~

    \begin{itemize}
        \item Basically, we have
              \begin{align*}
                  \mathrm{Pr}[\text{$C_j$ is satisfied}] & = 1 - \prod_{i\in S_j^+}(1-f(x_i^*))\prod_{i\in S_j^-}f(x_i^*)   \\
                                                         & \ge 1-\prod_{i\in S_j^+}4^{-x_i^*}\prod_{i\in S_j^-}4^{x_i^*-1}  \\
                                                         & = 1-4^{-(\sum_{i\in S_j^+}(x_i^*)+\sum_{i\in S_j^-}(1-x_i^*))} \\
                                                         & \ge 1-4^{-y_j^*}                                                 \\
                                                         & \ge (1-4^{-1})y_j^*
              \end{align*}
              where the last inequality is derived from the convexity of $4^{-x}$. Thus we obtain
              \begin{align*}
                  \frac{\mathbb{E}[\text{\# of satisfied clauses}]}{OPT} & \ge \frac{\sum\mathrm{Pr}[\text{$C_j$ is satisfied}]}{OPT_{LP}} \\
                                                                         & \ge \frac{\sum_{j=1}^{m}(1-4^{-1})y_j^*}{\sum_{j=1}^{m}y_j^*}   \\
                                                                         & = \frac34
              \end{align*}
              Then the approximation ratio is at least $\frac{3}{4}$
        \item we first proof the following lemma
              \begin{lemma}{*}
                  Suppose we have assigned the first i boolean variables $x_1 = a_1, \cdots, x_i = a_i$. Then we can compute
                  the expected value of solution
                  \[\mathbb{E}[W|x_1 = a_1, \cdots, x_i = a_i]\]
                  in polynomial time. Where $W=|\{{j|y_j = 1}\}|$
              \end{lemma}
              \begin{proof}
                  Observe that we can calculate the expectation of W conditioned on any partial set of assignments
                  to the variables: if a literal is false, then remove it from all the clauses in which it appears; if it is true, then
                  ignore the clauses which contain it, as they are already satisfied. Then the conditional expectation of W is the
                  unconditioned expectation of W in the reduced set of clauses plus the weight of the already satisfied clauses.
                  Thus let $f^{\prime}$ be the formula over variables $x_{i+1}, \cdots, x_n$ obtained from original formula $f$ by substituting values
                  of $x_1, \cdots, x_i$ and simply we can compute the expected value of $f^{\prime}$.
              \end{proof}
              Then we present the algorithm
              \begin{figure}[ht]
                  \centering
                  \begin{minipage}{.7\linewidth}
                      \begin{algorithm}[H]
                          \caption{Derandomized MAX-SAT}
                          \KwIn{Clauses $C_1,C_2, \cdots, C_m$.}
                          \KwOut{Boolean variable assignments $x_1, x_2, \cdots, x_n$}
                          \For{$i \gets 1$ \KwTo $n$}{
                              Compute $\mathbb{E}[W|x_1 = a_1, \cdots, x_{i-1} = a_{i-1}, x_i = True]$ and \\
                              $\mathbb{E}[W|x_1 = a_1, \cdots, x_{i-1} = a_{i-1}, x_i = False]$ \;
                              Pick the largest one and assign $x_i$ accordingly.
                          }
                      \end{algorithm}
                  \end{minipage}
              \end{figure}

              While we go through the setting of $x_1, x_2, \cdots, x_n$, the expectation is at least no lower than the all randomized version. So by this Derandomization, we get
              a better result than the original randomized version. Apply this to the algorithm using $f(x)$ as rounding function, we get a deterministic polynomial time algorithm
              with approximation ratio 3/4.

        \item If $f$ satisfies the form of $1-a^{-x} \le f(x) \le a^{x-1}$, $a>0$, then $a=4$ achieves the best approximation ratio.
              Let $g(x) = a^{x-1}-(1-a^{-x})$, $g(x)$ must be nonnegative, so $min g(x) \ge 0$. By doing derivation, we get $g^\prime(x) = a^{-1+x}\log a - a^{-x}\log a$,
              and $x_0 = \frac{1}{2}, g^\prime(x)=0$. So $a \le 4$, that $a=4$ achieve the best result we can hope.
    \end{itemize}
\end{solution}

\begin{solution}{4}
    ~

    \begin{itemize}
        \item The integer program for the problem is as follow
              \begin{equation*}
                  \begin{array}{rrclcl}
                      \displaystyle \textrm{minimize} & \multicolumn{3}{l}{\sum_{j=1}^m w_jx_j}                              \\
                      \\
                      \textrm{s.t.}                   & \displaystyle \sum_{u\in S_j}x_j        & \ge & 1, ~ \forall u \in U \\
                                                      & x_j                                     & \in & \{0, 1\}             \\
                  \end{array}
              \end{equation*}
              The LP relaxation version is
              \begin{equation*}
                \begin{array}{rrclcl}
                    \displaystyle \textrm{minimize} & \multicolumn{3}{l}{\sum_{j=1}^m w_jx_j}                              \\
                    \\
                    \textrm{s.t.}                   & \displaystyle \sum_{u\in S_j}x_j        & \ge & 1, ~ \forall u \in U \\
                                                    & x_j                                     & \in & [0, 1]
                                                    
                                                    \\
                \end{array}
            \end{equation*}
            \item Because the process applying to the variables rounded to 0 in previous iterations is independent to the ones rounded to 1, so it is the same
            to randomly round all variables and do the union $\mathcal{C}^\prime$ of all collection $\mathcal{C}$ of sets every turn. By one iteration, we can calculate the probability that an element
            $u \in U$ is covered. Suppose that $u$ occurs in $k$ sets of $U$. Let the probabilities being selected associated with these sets be $p_1, p_1, \cdots, p_k$. From 
            the LP constraintion, $p_1 + p_2 + \cdots + p_k \ge k$. Then
            \[\mathrm{Pr}[u \text{ is not coverd by } \mathcal{C}] = \prod_{i=1}^k (1-p_i) \le (1-\frac1k)^k \le \frac{1}{e}\]
            To get a complete set cover, independently pick $c\log n$ such subcollections, and compute their union, say $\mathcal{C}^\prime$. Then
            \[\mathrm{Pr}[u \text{ is not coverd by } \mathcal{C}^\prime] \le (\frac{1}{e})^{c\log n}\le(\frac{1}{n})^c\]
            Summing over all elements $u \in U$, we get
            \[\mathrm{Pr}[\mathcal{C}^\prime \text{ is not a valid set cover}] \le n \cdot (\frac{1}{n})^c = (\frac{1}{n})^{c-1}\]
            Clearly, $\mathbb{E}[weight(\mathcal{C}^\prime)] \le c \log n \cdot \mathbb{E}[weight(\mathcal{C})] \le c \log n \cdot OPT_{LP}$.  
            Applying Markov’s Inequality, we get
            \[\mathrm{Pr}[weight(\mathcal{C}^\prime) \ge \frac{c \log n \cdot OPT_{LP}}{(\frac{1}{n})^{c-1}}] \le (\frac{1}{n})^{c-1}\]
            The probability of the union of the two undesirable events is $\le 2(\frac{1}{n})^{c-1}$. Hence
            \[\mathrm{Pr}[\frac{SOL}{OPT}\le \frac{weight(\mathcal{C}^\prime)}{OPT_{LP}}=O(\log n) \text{ and } \mathcal{C}^\prime \text{ is a valid set cover}] \ge 1 - 2(\frac{1}{n})^{c-1} = 0.99\]
            where $c$ is selected to satisfy $1 - 2(\frac{1}{n})^{c-1} = 0.99$

            (reference from \textbf{Approximation Algorithms, Vazirani})
    \end{itemize}
\end{solution}

\begin{solution}{5}
    ~

    \begin{itemize}
        \item Suppose we have two objects, the first with size $s \le B$ and profit $100s$ and the second with size $B$ and profit $100B-1$, then
        ratio of first is slightly higher than the second. So the algorithm will choose the first item with much empty space unused.
        Hence, we have
        \[\frac{SOL}{OPT} = \frac{100s}{100B-1} \le 1, s \le B\]
        with $s$ approaches to 0, the approximation ratio will approaches to 0.
        \item We first ignore the case where all items fit in the bag, since this case is easy to test and gives an obvious result with respect to our problem.
        To hope for the most profit, if we can extend the capacity of the knapsack, we could fill it in order of profit ratio with increasing the size just enough
        to allow space for the next most profit-dense object. This would be an optimal profit for the slightly bigger knapsack. Thus
        \[OPT \le v_1 + v_2 + \cdots + v_k\]
        
        So, we have $\frac{OPT}{2} \le \max(v_k, \sum_{i=1}^{k-1}v_i)$, what achieves an approximation ratio of 1/2.
    \end{itemize}
    
\end{solution}
\end{document}
