%!BIB program = bibtex
\documentclass[12pt]{article}
 
\usepackage[margin=1in]{geometry} 
\usepackage{amsmath,amsthm,amssymb,mathtools}
\usepackage{enumerate}
\usepackage[ruled]{algorithm2e} %ruled vlined


 
\newcommand{\N}{\mathbb{N}}
\newcommand{\Z}{\mathbb{Z}}
 
\newenvironment{theorem}[2][Theorem]{\begin{trivlist}
    
\item[\hskip \labelsep {\bfseries #1}\hskip \labelsep {\bfseries #2.}]}{\end{trivlist}}
\newenvironment{lemma}[2][Lemma]{\begin{trivlist}
\item[\hskip \labelsep {\bfseries #1}\hskip \labelsep {\bfseries #2.}]}{\end{trivlist}}
\newenvironment{exercise}[2][Exercise]{\begin{trivlist}
\item[\hskip \labelsep {\bfseries #1}\hskip \labelsep {\bfseries #2.}]}{\end{trivlist}}
\newenvironment{problem}[2][Problem]{\begin{trivlist}
\item[\hskip \labelsep {\bfseries #1}\hskip \labelsep {\bfseries #2.}]}{\end{trivlist}}
\newenvironment{question}[2][Question]{\begin{trivlist}
\item[\hskip \labelsep {\bfseries #1}\hskip \labelsep {\bfseries #2.}]}{\end{trivlist}}
\newenvironment{corollary}[2][Corollary]{\begin{trivlist}
\item[\hskip \labelsep {\bfseries #1}\hskip \labelsep {\bfseries #2.}]}{\end{trivlist}}
\newenvironment{solution}[2][Solution]{\begin{trivlist}
\item[\hskip \labelsep {\bfseries #1}\hskip \labelsep {\bfseries #2.}]}{\end{trivlist}}
\begin{document}

% --------------------------------------------------------------
%                         Start here
% --------------------------------------------------------------

\title{Problem Set 1}%replace X with the appropriate number
\author{Chen Hao \\ MG20330007\\ %replace with your name
    Advanced Algorithms (Fall 2020)} %if necessary, replace with your course title

\maketitle

\begin{solution}{1} %You can use theorem, exercise, problem, or question here.  Modify x.yz to be whatever number you are proving
    The modified Karger's Contraction algorithm is as follow:

    \begin{algorithm}[H]
        % \LinesNumbered %enable line number
        \KwIn{an undirected weighted graph $G(V,E)$ with positive real weight} % input parameters
        \KwOut{a minimized cut $C$} % output result
        \While{$|V| > 2$}{
            choose an edge $uv \in E$ at random proportional to \textbf{weight}\;
            $G=Contract(G, uv)$\;
        }
        \KwRet{$C=\sum_{e \in E}w_e$}($E$ is the parallel edges between the only tow vertices in $V$)\;
        \caption{RandomContract2} % algotithm name
    \end{algorithm}
    To illustrate how that algorithm works, let's first proof the following lemma:
    \begin{lemma}{A}
        Given a fixed multigraph $G$ and a fixed minimum cut $C$, for every iteration:
        \begin{equation*}
            \text{Pr[randomly picked edge lies in $C$]} \le \frac{2}{n_i}
        \end{equation*}
        where $n_i$ denotes the number of vertices left after previous contraction($n_1=|V|=n$).
    \end{lemma}
    \begin{proof}
        Use notation $d(v)$ indicates the sum weight of edges incident to the vertex $v$, and $w_C$
        indicates the sum weight of edges in the cut $C$. Obviously, for any
        vertex $v$ of the $G$, we have that $d(v) \ge w_C$ (else it is smaller than the minimum cut, which is a
        contradiction). This implies that the total weights of $G$ satisfies $w_{Total} \ge n_iw_C/2$. Because the
        edge is picked proportional to its \textbf{weight}, we can have that:
        \begin{equation*}
            \text{Pr[randomly picked edge lies in $C$]} = \frac{w_C}{w_{Total}} \le \frac{w_C}{n_iw_C/2} = \frac{2}{n_i}
        \end{equation*}
    \end{proof}
    Returning to the algorithm, in order for the mimimum cut $C$ to be returned, the cut $C$ must
    survives all the $n-2$ iteration of selection. That is:
    \begin{align*}
        p_{correct} & = \text{Pr[a minimum cut is returned by RandomContract2]}                    \\
                    & \ge \text{Pr[$C$ is returned by RandomContract2]}                            \\
                    & = \text{Pr[$e_i \notin C$  for all i = 1,2,...,n-2]}                         \\
                    & = \text{$\prod_{i=1}^{n-2}$Pr[$e_i \notin C | \forall j < i, e_j \notin C$]} \\
                    & \ge \prod_{i=1}^{n-2}(1-\frac{2}{n-i+1})                                     \\
                    & = \frac{2}{n(n-1)}
    \end{align*}
    So the modified algorithm returns a weighted minimum cut with probability at least $\frac2{n(n-1)}$
\end{solution}

\begin{solution}{2}
    ~
    \begin{itemize}
        \item \begin{proof}
        for arbitray $\lambda \ge 0$, we have:
            \begin{align*}
                \text{Pr}[X \ge t] &= \text{Pr}[e^{\lambda X} \ge e^{\lambda t}] \\
                & \le \frac{\mathbb{E}[e^{\lambda X}]}{e^{\lambda t}} &\text{(Markov's inequality)} \\
                & = \text{exp}(-(\lambda t - \ln \mathbb{E}[e^{\lambda X}]))
            \end{align*}
        we can obtain a tight bound by select $\lambda$ to make $\lambda t - \ln E[e^{\lambda X}]$ 
        to get its supremum $\Psi_X^*(t) \coloneqq \sup_{\lambda \ge 0}(\lambda t - \Psi_X(\lambda))$.
        So $\text{Pr}[X \ge t] \le \text{exp}(-\Psi_X^*(t))$

        The remaining part of the proof is as folowed. We define $\Phi_X(\lambda) \coloneqq \lambda t
        - \Psi_X(\lambda), \lambda \ge 0$, then we have $\Psi_X^*(t) = \max_{\lambda \ge 0}\Phi_X(\lambda)$. 
        Apply derivation on $\Phi_X(\lambda)$ with respect to $\lambda$, 
        it comes that $\Phi_X^{\prime}(\lambda) = t - \Psi_X^{\prime}(\lambda)$. Because $\Phi_X(\lambda) = 
        \ln \mathbb{E}[e^{\lambda X}]$ is monotone increasing, if there exists a $\hat{\lambda}$ satisfies $\Psi_X^{\prime}(\hat{\lambda}) = t$,
        then it must be unique. In addition, $\Phi_X(\lambda)$ gets positive value at the left of $\hat{\lambda}$, and negative value the right, So
        $\Psi_X^*(t) = \max_{\lambda \ge 0}\Phi_X(\lambda) = \Psi_X(\hat{\lambda})$. If none of  $\lambda$ success, then the supreme can not be achieved.
        \end{proof}
        \item Normal random variables.
        \begin{align*}
            \Psi_X(\lambda) & = \ln \mathbb{E}[e^{\lambda X}] \\
            & = \ln (\frac1{\sqrt{2\pi}\sigma}\int e^{\lambda x}e^{-\frac{(x-\mu)^2}{2\sigma^2}} \text{d}x) \\
            & = \ln (\frac1{\sqrt{2\pi}\sigma}\int e^{-\frac{[x-(\mu + \sigma^2\lambda)]^2}{2\sigma^2} + \mu \lambda + \frac{\sigma^2 + \lambda}2} \text{d}x) \\
            & = \ln (e^{\mu \lambda + \frac{\sigma^2 + \lambda}2} \frac1{\sqrt{2\pi}\sigma}\int e^{-\frac{[x-(\mu + \sigma^2\lambda)]^2}{2\sigma^2}} \text{d}x) \\
            & = \mu \lambda + \frac{\sigma^2 + \lambda}2
        \end{align*}
        \begin{align*}
            \Psi_X^*(t) & = \sup_{\lambda \ge 0}(\lambda t - \Phi_X(\lambda)) \\
            & = \sup_{\lambda \ge 0}(\lambda t - \mu \lambda - \frac{\sigma^2 + \lambda}2) \\
            & = \max_{\lambda \ge 0}(\lambda t - \mu \lambda - \frac{\sigma^2 + \lambda}2)
        \end{align*}
        we define $\Phi_X(\lambda) = \lambda t - \Psi_X(\lambda) = \lambda t - \mu \lambda - \frac{\sigma^2 + \lambda}2 $ the same as above, and do derivation to obtain that:
        \[\Phi_X^{\prime}(\lambda) = t-\mu - \sigma^2\lambda\]
        let above equal to 0 to get the point when $\Phi_X(\lambda)$ achieve max value, thus we have:
        \begin{align*}
            \Psi_X^*(t) & = \max_{\lambda \ge 0}\Phi_X(\lambda) \\
            & = \Phi_X(\lambda)|_{\lambda=\frac{t-\mu}{\sigma^2}} \\
            & = \frac{(t-\mu)^2}{\sigma^2} - \frac{(t-\mu)^2}{2 \sigma^2} \\
            & = \frac{(t-\mu)^2}{2 \sigma^2}
        \end{align*}
        Otherwise, if the point cannot get, then $\Psi_X^*(t)=0$ \\
        In conclusion:
        \[\mathrm{Pr}[X \ge t] \le \mathrm{exp}({-\Psi_X^*(t)}) = \mathrm{exp}({\frac{(t-\mu)^2}{2 \sigma^2}})\]

        \item Poisson random variables.
        \begin{align*}
            \Psi_X(\lambda) & = \ln \mathbb{E}[e^{\lambda X}] \\
            & = \ln \sum_{k=0} \frac{e^{- v} v^k}{k!}e^{\lambda k} \\
            & = \ln(e^{-v}\sum_{k=0}\frac{(ve^\lambda)^k}{k!}) \\
            & = \ln(e^{-v}e^{ve^\lambda}) \\
            & = -v + ve^\lambda
        \end{align*}
        we define $\Phi_X(\lambda) = \lambda t - \Psi_X(\lambda) = \lambda t + v - ve^\lambda$ the same as above, and do derivation to obtain that:
        \[\Phi_X^{\prime}(\lambda) = t - ve^\lambda\]
        let above equal to 0 to get the point when $\Phi_X(\lambda)$ achieve max value, thus we have:
        \begin{align*}
            \Psi_X^*(t) & = \max_{\lambda \ge 0}\Phi_X(\lambda) \\
            & = \Phi_X(\lambda)|_{\lambda=\ln \frac{t}{v}} \\
            & = t \ln \frac{t}{v} + v -t
        \end{align*}
        Otherwise, if the point cannot get, then $\Psi_X^*(t)=v-1$ \\
        In conclusion:
        \[\mathrm{Pr}[X \ge t] \le \mathrm{exp}({-\Psi_X^*(t)}) = \mathrm{exp}(-t \ln \frac{t}{v} - v + t) = (\frac{t}{v})^{-t}e^{-v+t}\]
        \item Bernoulli random variables.
        \begin{align*}
            \Psi_X(\lambda) & = \ln \mathbb{E}[e^{\lambda X}] \\
            & = \ln (e^\lambda p + 1 \cdot (1-p))
        \end{align*}
        we define $\Phi_X(\lambda) = \lambda t - \Psi_X(\lambda) = \lambda t - \ln (e^\lambda p + 1 \cdot (1-p))$ the same as above, and do derivation to obtain that:
        \[\Phi_X^{\prime}(\lambda) = t - \frac{pe^\lambda}{pe^\lambda+(1-p)}\]
        let above equal to 0 to get the point when $\Phi_X(\lambda)$ achieve max value, thus we have:
        \begin{align*}
            \Psi_X^*(t) & = \max_{\lambda \ge 0}\Phi_X(\lambda) \\
            & = \Phi_X(\lambda)|_{\lambda=\ln \frac{t(1-p)}{p(1-t)}} \\
            & = t\ln \frac{t(1-p)}{p(1-t)} - \ln(t\frac{t(1-p)}{p(1-t)}p + (1-p)) \\
            & = -t\ln \frac{1-t}{1-p}+t\ln\frac{t}{p}-\ln\frac{1-p}{1-t} \\
            & = (1-t)\ln \frac{1-t}{1-p}+t\ln\frac{t}{p} \\
            & = D(X||Y)
        \end{align*}
        \item Sum of independent random variables.
        \begin{align*}
            \Psi_X(\lambda) & = \ln \mathbb{E}[e^{\lambda X}] \\
            & = \ln \mathbb{E}[\mathrm{exp}(\lambda \sum_{i=1}^n X_i)] \\
            & = \ln \prod_{i=1}^n \mathbb{E}[\mathrm{exp}(\lambda X_i)] \\
            & = \sum_{i=1}^n \ln \mathbb{E}[\mathrm{exp}(\lambda X_i)] \\
            & = \sum_{i=1}^n \Psi_{X_i}(\lambda)
        \end{align*}
        As $X_1, X_2, \cdots, X_n$ are independently and identically distributed random variables, $\Psi_{X_i}(\lambda) = \Psi_{X_j}(\lambda), \forall i \neq j$,
        What is to say:
        \begin{align*}
            \Psi_X^*(t) & = \sup_{\lambda \ge 0}[\lambda t - \Psi_X(\lambda)] \\
            & = \sup_{\lambda \ge 0}[\lambda t -\sum_{i=1}^n \Psi_{X_i}(\lambda)] \\
            & = \sup_{\lambda \ge 0}[\lambda t - n\Psi_{X_i}(\lambda)] \\
            & = \sup_{\lambda \ge 0}[n(\lambda \frac{t}{n} - \Psi_{X_i}(\lambda))] \\
            & = n \cdot \sup_{\lambda \ge 0}[\lambda \frac{t}{n} - \Psi_{X_i}(\lambda)] \\
            & = n\Psi_{X_i}^*(\frac{t}{n})
        \end{align*}
        For binomial random variable $X \sim \mathrm{B}(n,p)$, because $X = \sum_{i=1}^nX_i$, $X_1,X_2,\cdots,X_n$ is independently and identically distributed.
        From the above analysis, we have:
        \[\Psi_X(\lambda)= n\Psi_{X_i}(\lambda) = n \ln(e^\lambda p+(1-p))\]
        \[\Psi_X^*(t) =  n\Psi_{X_i}^*(\frac{t}{n}) = nD(Y_i||X_i)\]
        where $Y_i$ is a seris $\mathbf{i.i.d}$ Bernoulli random variable with parameter $\frac{t}{n}$. Thus:
        \[\mathrm{Pr}[X \ge t] \le \mathrm{exp}(-\Psi_X^*(t)) = \mathrm{exp}(-nD(Y_i||X_i))\]
        
        For random variable $X_i$ follows geometric distribution, we first calculate the $\Psi_{X_i}(\lambda)$ and $\Psi_{X_i}^*(\lambda)$:
        \begin{align*}
            \Psi_{X_i}(\lambda) & = \ln \mathbb{E}[e^{\lambda X_i}] \\
            & = \ln \sum_{k=1}e^{\lambda k}(1-p)^{k-1}p \\
            & = \ln pe^\lambda\sum_k[e^\lambda(1-p)^{k-1}] \\
            & = \ln pe^\lambda\frac{1}{1-e^\lambda(1-p)}
        \end{align*}
        we define $\Phi_{X_i(}\lambda) = \lambda t - \Psi_{X_i}(\lambda) = \lambda t - \ln pe^\lambda\frac{1}{1-e^\lambda(1-p)}$ the same as above, and do derivation to obtain that:
        \[\Phi_{X_i}^{\prime}(\lambda) = t - \frac{1}{1-e^\lambda(1-p)}\]
        let above equal to 0 to get the point when $\Phi_{X_i}(\lambda)$ achieve max value, thus we have:
        \begin{align*}
            \Psi_{X_i}^*(t) & = \max_{\lambda \ge 0}\Phi_{X_i}(\lambda) \\
            & = \Phi_{X_i}(\lambda)|_{\lambda = \frac{1-1/t}{1-p}} \\
            & = t \ln \frac{1-1/t}{1-p} - \ln p\frac{t-1}{1-p} \\
            & = (t-1)\ln\frac{t-1}{1-p} - t\ln t -\ln p
        \end{align*}
        then we obtain that:
        \[\Psi_{X}^*(t) = n\Psi_{X_i}^*(\frac{t}{n})=(t-n)\ln \frac{t/n-1}{1-p}-t\ln\frac{t}{n}-\ln p\]
        what follows:
        \[\mathrm{Pr}[X \ge t] \le \mathrm{exp}(-\Psi_X^*(t)) = (\frac{t/n-1}{1-p})^{(n-t)}+(\frac{t}{n})^t+p\]
    \end{itemize}
\end{solution}

\begin{solution}{3}
    ~\\
    Applying the taylor's expansion, that $e^x \sim 1+x$, we have:
    \[\mathrm{Pr}[e^{-\epsilon}Z \le \hat Z \le e^\epsilon Z] = \mathrm{Pr}[(1-\epsilon)Z \le \hat Z \le (1+\epsilon)Z] \ge 1-\delta\]
    what is to say:
    \[\mathrm{Pr}[|\hat Z - Z| \ge \epsilon Z] \le \delta\]
    From definition of $\hat Z$, we obtain that:
    \begin{align*}
        \mathbb{E}[\hat Z] & = \mathbb{E}[\prod_{i=1}^n \hat \rho_i] \\
        & = \prod_{i=1}^n\mathbb{E}[\hat \rho_i] \\
        & = \prod_{i=1}^n\mathbb{E}[\frac1s \sum_{j=1}^s X_i^{(j)}] \\
        & = \prod_{i=1}^n\frac1s \sum_{j=1}^s \mathbb{E}[X_i^{(j)}] \\
        & = \prod_{i=1}^n \rho_i \\
        & = Z
    \end{align*}
    Then, we have:
    \begin{align}
        \mathrm{Pr}[|\hat Z - Z| \ge \epsilon Z] & = \mathrm{Pr}[|\hat Z - \mathbb{E}[\hat Z]| \ge \epsilon Z] \\
        & \label{eqn2} \le \frac{\mathrm{Var}[\hat Z]}{\epsilon^2Z^2} & \text{(by Chebyshev's inequality)}
    \end{align}
    We now calculate the $\mathbf{Var}[\hat Z]$:
    \begin{align*}
        \mathbf{Var}[\hat Z] &= \mathbb{E}[\hat{Z}^2] - \mathbb{E}^2[\hat{Z}] \\
        & = \mathbb{E}[(\prod_{i=1}^n \hat \rho_i)^2] - (\prod_{i=1}^n \rho_i)^2 \\
        & = \mathbb{E}[\prod_{i=1}^n \hat \rho_i^2] - \prod_{i=1}^n \rho_i^2 \\
        & = \prod_{i=1}^n\mathbb{E}[ \hat \rho_i^2] - \prod_{i=1}^n \rho_i^2 \\
        & = \prod_{i=1}^n[\mathbf{Var}[\hat \rho_i] + \mathbb{E}^2[\hat \rho_i]] - \prod_{i=1}^n \rho_i^2 \\
    \end{align*}
    Because $X_i^{(j)}$ is $\mathbf{i.i.d}$, $\mathbf{Var}[\hat \rho_i] = \mathbf{Var}[\frac1s \sum_{j=1}^sX_i^{(j)}] = \frac1{s^2} \sum_{j=1}^s\mathbf{Var}[X_i^{(j)}] = \frac1s(1-\rho_i)\rho_i$.
    Togother with $\mathbb{E}[\hat \rho_i] = \mathbb{E}[\frac1s \sum_{j=1}^sX_i^{(j)}] = \sum_{j=1}^s\frac1s\mathbb{E}[ X_i^{(j)}] = \rho_i$, we have:
    \begin{align*}
        \mathbf{Var}[\hat Z] & = \prod_{i=1}^n[\mathbf{Var}[\hat \rho_i] + \mathbb{E}^2[\hat \rho_i]] - \prod_{i=1}^n \rho_i^2 \\
        & = \prod_{i=1}^n[\frac1s(1-\rho_i)\rho_i + \rho_i^2] - \prod_{i=1}^n \rho_i^2
    \end{align*}
    Back to~\eqref{eqn2}
    \begin{align}
        \mathrm{Pr}[|\hat Z - Z| \ge \epsilon Z] & \le \frac{\mathrm{Var}[\hat Z]}{\epsilon^2Z^2} \\
        & =  \frac{1}{\epsilon^2} \cdot \frac{\prod_{i=1}^n[\frac1s(1-\rho_i)\rho_i + \rho_i^2] - \prod_{i=1}^n \rho_i^2}{\prod_{i=1}^n \rho_i^2} \\
        & = \frac{1}{\epsilon^2} \cdot [\frac{\prod_{i=1}^n[\frac1s(1-\rho_i)\rho_i + \rho_i^2]}{\prod_{i=1}^n \rho_i^2} -1]\\
        & = \label{eqn3}\frac{1}{\epsilon^2} \cdot [\prod_{i=1}^n[\frac1s(\frac1\rho_i-1) + 1] -1]
    \end{align}
    To give~\eqref{eqn3} a tight bound $\mathrm{Pr}[|\hat Z - Z| \le \delta$, we observe that $\rho_i \ge \frac12$ and have that:
    \begin{gather*}
        \frac{1}{\epsilon^2} \cdot [\prod_{i=1}^n[\frac1s(\frac1\rho_i-1) + 1] -1]|_{\rho_i = \frac12} = \delta \\
        \frac{1}{\epsilon^2}[(\frac1s+1)^n-1] = \delta \\
        (1 + \frac1s)^n = \epsilon^2\delta + 1 \\\
        s = \frac{1}{(\epsilon^2\delta + 1)^{1/n}-1}
    \end{gather*}
\end{solution}

\begin{solution}{4}
    ~\\
    Suppose there exist 3 set of bins, $\mathcal{A}, \mathcal{B}$ and $\mathcal{C}$, all have $n$ empty bins at the beginning. There
    are four ways to drop a ball:
    \begin{enumerate}[(a)]
        \item select a bin of number $i \in [n]$ independently and uniformly at random and drop a ball in;
        \item select two bins of number $i \in [n]$ independently and uniformly at random and choose the least loaded to drop a ball;
        \item select two bins of number $i \in [n]$ independently and uniformly at random and drop nothing;
        \item select two bins of number $i \in [n]$ independently and uniformly at random and choose the first to drop a ball;
    \end{enumerate}

    For every round of dropping balls, if method (a) is appiled to $\mathcal{A}$, then do the same with $\mathcal{B}$ and $\mathcal{C}$.
    Otherwise, if method (b) is appiled to $\mathcal{A}$, then drop ball with method (c) to $\mathcal{B}$ and with method (d) to $\mathcal{C}$. 
    The count of appling method (a) is the same of method (b), that is $n/2$

    Let function $\mathbf{M}(S)$ denotes the number of balls in max-loaded bin of set $S$ with high probability.
    For coupling set $(\mathcal{A}, \mathcal{B}, \mathcal{C})$, since the ball number in max-loaded bins is not increasing from method
    (d) to (b) to (c), we have $\mathbf{M}(\mathcal{B}) \le \mathbf{M}(\mathcal{A}) \le \mathbf{M}(\mathcal{C})$.

    For $\mathcal{B}$, we only drop half of the $n$ balls, which is $\Theta(n)$, so $\mathbf{M}(\mathcal{B}) = \Theta(\frac{\log n}{\log \log n})$

    For $\mathcal{C}$, we do the same as appling method (a) $n$ times, which is $\Theta(n)$, so $\mathbf{M}(\mathcal{C}) = \Theta(\frac{\log n}{\log \log n})$

    Follow Sandwich Theorem, we have that $\mathbf{M}(\mathcal{A}) = \Theta(\frac{\log n}{\log \log n})$.

    Because the distribution of sololy dropping balls in $\mathcal{A}$ is the same as the marginal distribution of $\mathcal{A}$ when dropping
    balls to coupling set $(\mathcal{A}, \mathcal{B}, \mathcal{C})$. So they have the same expectation value under operation of $\mathbf{M}$.
    For all the three paradigms in the question, as the discution above, the order of appling method (a) or (b) have nothing to do with the max load.
    In conclusion, the maximum load with high probability is all $\Theta(\frac{\log n}{\log \log n})$ in three paradigms.
\end{solution}

\begin{solution}{5}
    ~
    \begin{enumerate}
        \item If $g \equiv 0$, then for arbitray $\vec{x}$, there exists $k$ that $a_kx_k-b_k = 0 \mod{p}$. Since
        $0 \le a_k, b_k \le n < p$, then there must be $x_k = 0 \mod{p}$ or $a_k = b_k = 0$. However, the $\vec{x}$
        is arbitraily selected, so there we have $f(\vec{x}) = (a_kx_k - b_k)\prod_{i \neq k}(a_ix_i-b_i) \equiv 0$.
        That is to say $f \not \equiv 0 \implies g \not \equiv 0$.
        \item\begin{algorithm}[h]
            choose $r_1, r_2, \cdots, r_n \in \mathbb{Z}_p$ uniformly and independently at random\;
            if $g(\vec r) = g(r_1, r_2, \cdots, r_n) = 0$ then return "yes" else return "no" \;
            \caption{Randomized algorithm for multivariate PIT} % algotithm name
        \end{algorithm}
        if $g \equiv 0$, the algorithm always returns "yes", from the proof above, it is always correct. \\
        if $g \not \equiv 0$, the algorithm may wrongly return "yes" (a false positive). But this happens only when the random
        $\vec r = (r_1, r_2, \cdots, r_n)$ is a root of $g$. The probability of this bad event is upper bounded by Schwartz-Zippel Theorem
        as that,
        \[\mathrm{Pr}[g(r_1, r_2, \cdots, r_n) = 0] \le \frac{n}{p} = \epsilon\]
        where $n$ is the degree of the nonzero $n$-variate polynomial $g$ which is strictly small than $p$.
    \end{enumerate}
\end{solution}

\begin{solution}{6}
    ~
    \begin{enumerate}[(a)]
        \item \begin{proof}
            We only prove this for discrete distrution by induction on the number
            of mass point. The continuous condtion can be extended by replacing
            continuity argument. For tow-mass-point distribution, the inequality becomes
            \[p_1f(x_1) + p_2f(x_2) \ge f(p_1x_1+p_2x_2)\]
            where $p_1+p_2=1, p_1,p_2 \ge 0$. It follows directly from the convex
            function definition.
            Assuming that the inequality is true for distribution with k-1 mass points.Then
            for k, we have:
            \begin{align*}
                \mathbb{E}[f(X)] & = \sum_{i=1}^{k}p_if(x_i)                                                                                                                  \\
                                 & = p_kf(x_k)+\sum_{i=1}^{k-1}p_if(x_i)                   & (\text{rewrite $p_i=p_i^{\prime}*(1-p_k)$})                                      \\
                                 & = p_kf(x_k)+(1-p_k)\sum_{i=1}^{k-1}p_i^{\prime}f(x_i)   & (\text{since $\sum_{i=1}^{k-1}p_i^{\prime}=\frac1{1-p_k}\sum_{i=1}^{k-1}p_i$}=1) \\
                                 & \ge p_kf(x_k)+(1-p_k)f(\sum_{i=1}^{k-1}p_i^{\prime}x_i)                                                                                    \\
                                 & \ge f(p_kx_k+(1-p_k)\sum_{i=1}^{k-1}p_i^{\prime}x_i)                                                                                       \\
                                 & = f(\sum_{i=1}^{k}p_ix_i)                                                                                                                  \\
                                 & = f(\mathbb{E}[X])
            \end{align*}
        \end{proof}
        \item \begin{proof}
            The funtion $f(x)=x$ is strictly convex, as $f^{\prime \prime}(x)=\frac1t\log e > 0$ for all positive x. 
            Thus by Jensen's inequality, we have:
            \[\sum\alpha_i f(x_i) \ge f(\sum\alpha_i x_i)\]
            where $\alpha_i \ge 0, \sum_i \alpha_i = 1$. Let $\alpha_i = \frac{b_i}{\sum_{j=1}^{n}{b_j}}$ and $x_i=\frac{a_i}{b_i}$, we obtain:
            \[\sum(\frac{b_i}{\sum b_j}\times \frac{a_i}{b_i}\log \frac{a_i}{b_i}) \ge (\sum\frac{b_i}{\sum b_j}\frac{a_i}{b_i})\log (\sum\frac{b_i}{\sum b_j}\frac{a_i}{b_i})\]
            Simplify it, we get:
            \[\sum(\frac{a_i}{\sum b_j}\log \frac{a_i}{b_i}) \ge (\sum\frac{a_i}{\sum b_j})\log (\sum\frac{a_i}{\sum b_j})\]
            which is the log sum inequality.
        \end{proof}
        \item \begin{proof}
        Let $Y_i = X_i-\mathbb{E}[X_i]$, and $Y = \sum_{i=1}^nY_i = X-\mathbb{E}[X]$, thus $\mathbb{E}[Y_i]=0$, and $\mathbb{E}[Y]=0$.
        Let $Y_i^{\prime}$ denote an independent copy of $Y_i$ then we have that,
        \[\mathbb{E}_{Y_i}[\mathrm{exp}(\lambda Y_i)] = \mathbb{E}_{Y_i}[\mathrm{exp}(\lambda(Y_i-\mathbb{E}[Y_i^{\prime}])] \le \mathbb{E}_{Y_i,Y_i^\prime}[\mathrm{exp}(\lambda(Y_i-Y_i^{\prime})]\]
        using Jensen's inequality, and the convexity of the funtion $\mathrm{exp}(x)$.
        Now, let $\epsilon$ be a Rademacher random variable. Then note that the distribution of $Y_i - Y_i^\prime$
        is identical to the distribution of $\epsilon(Y_i- Y_i^\prime)$ So we obtain that,
        \begin{align*}
            \mathbb{E}_{Y_i,Y_i^\prime}[\mathrm{exp}(\lambda(Y_i-Y_i^{\prime})] & = \mathbb{E}_{Y_i,Y_i^\prime}[\mathbb{E}_\epsilon[\mathrm{exp}(\lambda\epsilon(Y_i-Y_i^{\prime})]] \\
            & \le \mathbb{E}_{Y_i,Y_i^\prime}[\mathrm{exp}(\lambda^2(Y_i-Y_i^{\prime})^2/2]
        \end{align*}
        where we use the hint with $(Y_i-Y_i^{\prime})$ fixed by condition. Now $(Y_i-Y_i^{\prime})$ using boundedness is at most $(b-a)$, so we obtain that,
        \[\mathbb{E}_{Y_i}[\mathrm{exp}(\lambda Y_i)] \le \mathrm{exp}(\lambda^2(b-a)^2/2)\]
        Then we have:
        \begin{align*}
            \mathrm{Pr}[Y \ge t] & = \mathrm{Pr}[e^{\lambda Y} \ge e^{\lambda t}] \\
            & \le \frac{\mathbb{E}[e^{\lambda Y}]}{e^{\lambda t}} \\
            & = e^{-\lambda t}\mathbb{E}[\prod_{i=1}^n e^{\lambda Y_i}] \\
            & = e^{-\lambda t}\prod_{i=1}^n \mathbb{E}[e^{\lambda Y_i}] \\
            & \le e^{-\lambda t}\prod_{i=1}^n exp(\lambda^2(b-a)^2/2) \\
            & = e^{-\lambda t} e^{n\lambda^2(b-a)^2/2}
        \end{align*}
        Minimizing over $\lambda$, we get
        \[\mathrm{Pr}[Y \ge t] \le exp(-\frac{t^2}{2n(b-a)^2})\]       
        Repeating this in the other direction we get
        \[\mathrm{Pr}[|Y| \ge t] \le 2exp(-\frac{t^2}{2n(b-a)^2})\]
        using the union bound.
        In conclusion, we have,
        \[\mathrm{Pr}[|X-\mu| \ge t] \le 2exp(-\frac{t^2}{2n(b-a)^2})\]
        \end{proof}
    \end{enumerate}
\end{solution}

\end{document}