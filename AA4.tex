%!BIB program = bibtex
\documentclass[12pt]{article}
 
\usepackage[margin=1in]{geometry} 
\usepackage{amsmath,amsthm,amssymb,mathtools}
\usepackage{enumerate}
\usepackage[vlined, ]{algorithm2e} %ruled vlined

 
\newcommand{\N}{\mathbb{N}}
\newcommand{\Z}{\mathbb{Z}}
\renewcommand{\thealgocf}{}

\newenvironment{theorem}[2][Theorem]{\begin{trivlist}
    
\item[\hskip \labelsep {\bfseries #1}\hskip \labelsep {\bfseries #2.}]}{\end{trivlist}}
\newenvironment{lemma}[2][Lemma]{\begin{trivlist}
\item[\hskip \labelsep {\bfseries #1}\hskip \labelsep {\bfseries #2.}]}{\end{trivlist}}
\newenvironment{exercise}[2][Exercise]{\begin{trivlist}
\item[\hskip \labelsep {\bfseries #1}\hskip \labelsep {\bfseries #2.}]}{\end{trivlist}}
\newenvironment{problem}[2][Problem]{\begin{trivlist}
\item[\hskip \labelsep {\bfseries #1}\hskip \labelsep {\bfseries #2.}]}{\end{trivlist}}
\newenvironment{question}[2][Question]{\begin{trivlist}
\item[\hskip \labelsep {\bfseries #1}\hskip \labelsep {\bfseries #2.}]}{\end{trivlist}}
\newenvironment{corollary}[2][Corollary]{\begin{trivlist}
\item[\hskip \labelsep {\bfseries #1}\hskip \labelsep {\bfseries #2.}]}{\end{trivlist}}
\newenvironment{solution}[2][Solution]{\begin{trivlist}
\item[\hskip \labelsep {\bfseries #1}\hskip \labelsep {\bfseries #2.}]}{\end{trivlist}}
\begin{document}

% --------------------------------------------------------------
%                         Start here
% --------------------------------------------------------------

\title{Problem Set 4}%replace X with the appropriate number
\author{Chen Hao \\ MG20330007\\ %replace with your name
    Advanced Algorithms (Fall 2020)} %if necessary, replace with your course title

\maketitle

\begin{solution}{1}
    ~

    \begin{itemize}
        \item Becasue $T$ will not choose vertices from the same subset, we only take into concern edges across
        two distinct sets. Suppose we pick up from each subset $S_i$, randomly and independently, a unique vertex
        according to a uniform distrubution, that is, in each $S_i$ the probability of a vertex to be chosen is $1/|S_i|$.

        Let $E^* = \{e = \{u,v\} | ~ |\{u, v\} \cap S_i| \le 1, \forall i = 1, \cdots, r\}$, contains all edges whose end-points
        are contained in two distinct subsets. For $e = (u, v) \in E^*$, $u \in S_i, v \in S_j$, let $A_e$ be the bad event $\{u, v\} \subset T$. Then
        from $S_i$ and $S_j$ randomly selecting, the probability bad event $A_e$ happen is 
        $\mathrm{Pr}[A_e] = \frac{1}{|S_i|\cdot |S_j|} \le \frac{1}{(2e\Delta)^2} = p$. The bad event is mutually independent of all 
        the other bad events involving edges whose endpoints do not lie in $S_i \cup S_j$.

        So the graph $H$ with vertex set $V_H = E^*$, and edge set
        \[E_H = \{(e, e^\prime)|~ |e \cap S_i| + |e^\prime \cap S_i| > 1, k = 1, \cdots, r\}\]
        is a dependency graph for the bad events $\{A_e\}_{e\in E^*}$. Since a subset $S_i$ contains $\Delta\cdot|S_i|$ cross edges, a
        edge $e$ could at most share endpoint with $2(\Delta\cdot|S_i|-1)$ edges. So the degree of the graph $d \le 2(2e\Delta)-2$.

        Hence, $p\cdot e(d+1) \le \frac{1}{(2e\Delta)^2}(4e^2\Delta^2) \le 1$.
        Applying Lovász Local Lemma (symmetric case), we have
        \[\mathrm{Pr}[\cap_{e\in E^*}\overline{A_e}] > 0\]
        So there must be an independent transversal by the probabilistic method.
        (reference from \textbf{An Improvement of the Lov´asz Local Lemma via Cluster Expansion, Rodrigo Bissacot et al.})
        \item The algorithm is as follow, obtained by fixing a little the Moser-Tardos's one
        \\[60pt]
        \begin{figure}[ht]
            \centering
            \begin{minipage}{.7\linewidth}
        \begin{algorithm}[H]
            \caption{greedy maximum coverage}
            \KwIn{pattition $S_1,S_2,\cdots,S_r$}
            i.i.d select a vertex from each subset $S_1, \cdots, S_r$, get $v_1, \cdots, v_r$ \;
            \While{$\exists$ bad event $A_e$ happen, $e = \{u, v\}, u \in S_i, v \in S_j$}{
                reselect two vertex seperately in $S_i$ and $S_j$
            }
            \Return{maxcover}
        \end{algorithm}
    \end{minipage}
\end{figure}

        From LLL, we could get an independent transversal in $O(r+2m/d) = O(r+|E|/2e^2\Delta^2)$
    \end{itemize}
\end{solution}

\begin{solution}{2}
    ~

    \begin{itemize}
        \item We choose i.i.d. each vertex from V with probability $p$ to construct set $X$. Let $Y$ to be the 
        set of vertices that are not in $X$ and are not neighbors to any vertex in $X$ (namely, whose neighbors are not in $X$). 
        Thus $D=X \oplus Y (X \cap Y = \emptyset)$ is a dominating set.

        We have $\mathbb{E}[|X|]=np$. Addtionaly, for fixed $v \in Y$,
        \[\mathrm{Pr}[v \in Y] = \mathrm{Pr}[v \not \in X, neighbor(v) \not \in X]\le (1-p)^{d+1}\]
        So $\mathbb{E}[|Y|] = \mathbb{E}[\sum_{i=1}^n \mathbb{I}[v_i \in Y]]= \sum_{i=1}^n\mathbb{E}[ \mathbb{I}[v_i \in Y]]
        =\sum_{i=1}^n\mathrm{Pr}[v_i \in Y] \le n(1-p)^{d+1}$. Combinate above up, we could result in
        \[\mathbb{E}[|D|] = \mathbb{E}[|X|+|Y|] = \mathbb{E}[|X|] + \mathbb{E}[|Y|] \le np + n(1-p)^{d+1} \le np + ne^{-p(d+1)}\]
        where the last inequality derived from $1-p \le e^{-p}$. Finally, we optimize the value of $p$ to get a optimal bound.
        
        With $p=\frac{\ln (d+1)}{d+1}$, we have
        \begin{align*}
            \mathbb{E}[|D|] & \le n\frac{\ln (d+1)}{d+1} + ne^{-\frac{\ln (d+1)}{d+1}(d+1)} \\
            & = \frac{n(1+\ln (d+1))}{d+1}
        \end{align*}
        So there exists a dominating set with size at most $\frac{n(1+\ln (d+1))}{d+1}$, otherwise $\mathbb{E}[|D|]$ will contradict
        to above inequality.

        \item Let $D \subseteq V$ be a subset of vertices obtained by tossing a coin for each vertex $v \in V$ independently and randomly with
        probability $p = \frac{\ln(2e(d^2+1))}{2d}$. 

        For each vertex $v \in V$, let $A_v$ be the bad event $v$ is not dominated by $D$. Events $A_u, A_v$ are dependent if and only
        if their outcomes depend on at least one common coin toss, namely, $neighbor(u) \cap neighbor(v) = \emptyset$. Hence, the degree of
        the dependency graph is at most $d^2$.
        \[\mathrm{Pr}[A_v]=(1-p)^{d(v)} = e^{d(v)\ln (1-p)} \le e^{-d(v)p}\le \frac{1}{2e(d^2+1)}\]
        Applying the Local Lemma, we have
        \[\mathrm{Pr}[\bigcap_{v\in V}\overline{A_v}] > (1-\frac{1}{d^2+1}) > \exp(-\frac{n}{d^2+1})\]
        Let $X$ be the size of dominating set, then
        \[\mathrm{Pr}[(x < (1+\epsilon)np) \bigcap (\bigcap_{v\in V}\overline{A_v})]> 0 \Leftrightarrow \mathrm{Pr}[\bigcap_{v\in V}\overline{A_v}] - \mathrm{Pr}[x \ge (1+\epsilon)np] > 0\]
        From Chernoff bound, we have
        \[\mathrm{Pr}[x \ge (1+\epsilon)np] \le \exp(-\frac{\epsilon^2np}{3})\]
        By optimize $\epsilon$, we have an upper bound of $2n\ln d /d$, what is worse when $d > 2$.
    \end{itemize}
\end{solution}

\end{document}